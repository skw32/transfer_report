\chapter{Methodology}

\section{Calculation of the Formation Energy of Defects in \CZTS}
See Lany paper + Keith's posted paper + encyclopoedia of defects paper 


\subsection{Density Functional Theory \& Hybrid Functionals}\label{DFT_section}
See Bechstedt book \cite{Bechstedt} + figures from Aron + discuss band gap underestimation in standard DFT (GGA mentioned during intro, also discuss GGA+U and limitations)

\subsection{The Supercell Method \& Corrections for Charged Defects}\label{supercell_section}
see Aron's lectures + DFT in materials science paper \cite{DFT_in_mat}


\subsection{Defect Formation Energy \& Equilibrium Concentration}
Show formula for defect formation E as a function of electronic and atomic chemical potentials, discuss chemical potentials (especially S!) and show Boltzmann expression for concentration.



\section{Monte Carlo Simulation of Thermodynamic Disorder in \CZTS}
\begin{itemize}
\item Mapping crystal structure to a lattice
\item General MC simulations and Metropolis algorithm
\item Ising model and likeness of our simulation to ising model
\end{itemize}

\subsection{Multi-Scale Approach}
DFT scaling: correcting bulk/ macroscopic dielectric constant with DFT to account for dielectric screening (HSE vs gulp plot), state using [CuZn + ZnCu] formation energies calculated during MRes project

\subsection{Quantification of Disorder Using Radial Distribution Functions}
\begin{itemize}
\item Reproducing GS
\item Cu-Zn RDF first peak to describe Cu Zn Cu layer
\item Cu-Sn RDF firt peak to describeCu Sn Cu layer
\end{itemize}


\subsection{Band Tailing from the Distribution of Electrostatic Potential}


\section{Calculation of Intrinsic Band Gap Broadening in \CZTS}


\section{Calculation of Optoelectronic Properties of Candidate Solar Absorber Materials}

                            