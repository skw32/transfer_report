
\chapter{Background Theory}

\label{ch:background}

\section{The Description of Perfect Periodic Crystal Structures}

\section{Crystal Imperfections of the First and Second Types}
Notes from A. Guinier 'X-Ray Diffraction': use Ch6 to introduce then Ch8 for long and short range order in mixed crystals with substitutional disorder + see Ziman

\section{Band Theory \& Band Structure}\label{band_theory}
See MRes2 report + use e.g. of Si, state that early figure is a simplification and use to discuss why Si is not ideal from indirect band gap, use rough calc from lecture

\section{Spin Orbit Interaction}
 see webpages: pg 84 for discussion of effect of SOC on lattice without inversion symmetry!  + useful slide

Look for textbook source?

\section{Photoluminescence Spectra of Solar Absorber Materials}\label{PL_section}
Photoluminescence (PL) imaging is becoming a popular method to inspect solar cell materials, it does not require a full functioning device and can be a powerful tool for probing defects in semiconductors \cite{characterization_book, Gerschon}. The PL spectra of \CZTS (CZTS) provides clear evidence of disorder in the material.

Overview of technique, information gained from technique, T dependent PL, PL spectra of CZTS: single crystal and thin film.

\section{Band Tailing in Disordered Semiconductors}
Nelson, pg 65, 3.5.4: heavy doping leading to band tailing

see Pankove + Russian 1970s papers, Urbach tail, fluctuations in electrostatic potential\\
See Urbach tail doc and use Cu/ Zn culprit paper

\section{Impact of Defects and Disorder on Photovoltaic Performance}\label{defects_in_PV}
SRH recomb, GBs, secondary phases, band gap and electrostatic potential fluctuations\\
See CMP lectures on defects + ebook reading material 

\section{Photovoltaic-Ferroelectric Phenomena \& the Possibility of High Performance Solar Cells}\label{FE_PV_section}

Ferroelectric PV materials are currently receiving a great deal of research interest, however the origin of their PV properties are considered to be unresolved \cite{Rappe}. A large number of theories have been proposed in an attempt to explain the two observed ferroelectric-photovoltaic (FE-PV) phenomena: the bulk PV effect (BPE), also referred to as the photogalvanic effect, and the anomalous PV effect (APE). In the BPE, a direct current appears in a homogeneous medium under uniform illumination and this can occur in all materials without a center of symmetry  \cite{PGE}. Ferroelectric materials exhibit this effect strongly \cite{Rappe} and the first observation of this effect was in 1956 with photovoltages measured in un-doped single crystals of the ferroelectric material BaTiO$_3$ \cite{keith_46}. In the case of the APE, photovoltages have been measured that are orders of magnitude larger than the band gap of the material \cite{keith_54}, but has been observed to disappear when the sample undergoes a phase transition to a paraelectric phase \cite{nonlinear_dielectric}, and so no longer exhibits spontaneous electric polarization. Theories have been developed to explain the FE-PV phenomena based around experimental observations of factors that have been shown to influence the photovoltage of FE-PV devices, such as:  the distance between the two opposite electrodes \cite{rev_28,rev_46}, intensity of incident light \cite{rev_47}, electrical conductivity \cite{Fridkin}, remnant polarization of the
ferroelectric crystals \cite{rev_48}, crystallographic orientation \cite{rev_49}, the dimension or size of the crystals \cite{rev_46, rev_50}, domain walls \cite{rev_30}
and the interface between the FE material and the electrode \cite{rev_37}.\\

Models have been proposed to explain the BPE in ferroelectric materials based upon the built-in asymmetry of non-centrosymmetric crystals. One model is based on asymmetric scattering centres in the materials \cite{PGE}. In non-centrosymmetric crystals, the rate of the generation of charge carriers with momenta $\pm$k can be different due to asymmetric electron-hole scattering. A `ballistic current' can then be generated due to the momentum imbalance \cite{shift_current}.
Another model, the shift current model \cite{shift_current}, has been proposed, which is based on the asymmetry of the electron density \cite{keith}.
Light-induced transitions of charge carriers between bands in reciprocal space are accompanied by asymmetrical shifts in real space between atoms in elementary cells \cite{shift_current}.
Such currents have been demonstrated for a number of materials, such as GaAs \cite{keith_52} and BiFeO$_3$, where this has been demonstrated using both computational \cite{keith_25} and experimental \cite{keith_51} techniques. The shift current model has been used in first principles calculations to reproduce experimental photocurrent direction and magnitude as a function of light frequency resulting from the BPE in BaTiO$_3$\cite{Rappe}. 
The nonlinear dielectric model has also been proposed to explain the BFE effect in (Pb, La)(Zr, Ti)O$_{3}$ (PLZT) ceramics \cite{nonlinear_dielectric}. This material exhibits the photostrictive effect, where strain is induced in the sample by incident light. In this model, the large photovoltage is believed to be due to the nonlinear response of the material to the incident light, which results in an effective DC electric field throughout the ferroelectric material \cite{FE_PV_rev1}.\\

The domain wall theory has been proposed to explain the large generated photovoltages in the APE \cite{domain_wall} and the Schottky-junction effect \cite{schottky_effect} and depolarization field model \cite{screen_effect, depol_model}, also referred to as the screening effect, have been proposed as additional contributions to the large photovoltage. Unlike the BPE, some theories to explain the APE rely on the nano- and microstructure of the material \cite{keith}. The latter two theories are related to the interface between the FE material and an electrode in a FE-PV device, but were originally neglected as the contributions to the photovoltage were believed to be small. However, these effects become more significant in thin-film devices where photovoltages are typically low \cite{FE_PV_rev1}, and thin-films are particularly relevant for PV applications.
The domain wall theory was developed to explain observations of photovoltages in thin films of BiFeO$_3$ increasing linearly with the total number of ferroelectric domain walls along the net direction of electric polarization and vanishing along the direction perpendicular to the net polarization \cite{rev_30}. In this theory, the narrow ferroelectric domain walls drive the dissociation of photogenerated excitons and so  act as nanoscale photovoltage generators connected in series. The photocurrent across the domain walls is therefore continuous but the photogenerated voltage accumulates along the direction of net polarization, allowing for photovoltages that are considerably larger than the band gap of the material \cite{FE_PV_rev1}.
In the Schottky-junction effect, the FE semiconductor forms a Schottky contact with the metal electrodes, which then generate a photocurrent under illumination due to the local electric field caused by the band bending near to the electrode. This photocurrent is dependent upon the Schottky barrier height and depletion region depth, but the photovoltage is still limited to the band gap of the material. Further, the additional photovoltage contribution from this effect can be cancelled out if the same electrode contacts are used, due to the opposite polarization of the two Schottky-junctions \cite{FE_PV_rev1}.
In the depolarization field model, high densities of polarization charges are believed to accumulate on surfaces of polarized FE films, this then induces a large electric field inside the FE layer if the charge is not screened. This effect will be far more pronounced in a thin-film device. The electric field is thought to not be fully screened by the free charges in the metal or semiconductor that the FE layer is in contact with, resulting in a depolarization field. This depolarization field will be larger when the FE material has a large remnant electric polarization, the FE layer is thinner and when it is in contact with a semiconductor, as opposed to a metal, due to fewer free charge carriers and higher dielectric constant in a semiconductor than a metal, giving weaker screening. The depolarization field is believed to be the dominating force for the separation of photogenerated charge carrier pairs \cite{FE_PV_rev1}.