
\chapter{Background Theory}

\label{ch:background}

\section{The Description of Perfect Periodic Crystal Structures}

\section{Crystal Imperfections of the First and Second Types}
Notes from A. Guinier 'X-Ray Diffraction': use Ch6 to introduce then Ch8 for long and short range order in mixed crystals with substitutional disorder + see Ziman

\section{Band Theory \& Band Structure}\label{band_theory}
See MRes2 report + use e.g. of Si, state that early figure is a simplification and use to discuss why Si is not ideal from indirect band gap, use rough calc from lecture

\section{Spin Orbit Interaction}
 see webpages: pg 84 for discussion of effect of SOC on lattice without inversion symmetry!  + useful slide

Look for textbook source?

\section{Photoluminescence Spectra of Solar Absorber Materials}\label{PL_section}
Photoluminescence (PL) imaging is becoming a popular method to inspect solar cell materials, it does not require a full functioning device and can be a powerful tool for probing defects in semiconductors \cite{characterization_book, Gerschon}. The PL spectra of \CZTS (CZTS) provides clear evidence of disorder in the material.

Overview of technique, information gained from technique, T dependent PL, PL spectra of CZTS: single crystal and thin film.

\section{Band Tailing in Disordered Semiconductors}
Nelson, pg 65, 3.5.4: heavy doping leading to band tailing

see Pankove + Russian 1970s papers, Urbach tail, fluctuations in electrostatic potential\\
See Urbach tail doc and use Cu/ Zn culprit paper

\section{Impact of Defects and Disorder on Photovoltaic Performance}\label{defects_in_PV}
SRH recomb, GBs, secondary phases, band gap and electrostatic potential fluctuations\\
See CMP lectures on defects + ebook reading material 

\section{Photovoltaic-Ferroelectric Phenomena \& the Possibility of High Performance Solar Cells}