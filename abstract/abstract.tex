
\addcontentsline{toc}{chapter}{Abstract}

\begin{abstract}

Due to the large energy payback time of silicon solar cells there is a strong drive to develop new solar absorber materials for cost-effective and, ideally, highly-efficient solar cells. A number of metal sulfide materials have been found to have some desirable optical properties for this application, such as direct and sunlight-matched band gaps. Furthermore, many of these materials have the additional benefit over other solar cell technologies such as CdTe and CuIn$_x$Ga$_{1-x}$Se$_2$ (CIGS) of containing only earth-abundant and non-toxic components. {\CZTS} (CZTS) has received a great deal of scientific interest in recent years for this reason. With a direct band gap of 1.5 eV, the Shockley-Quiesser limit would predict a theoretical power conversion efficiency (PCE) limit for a CZTS solar cell of 28\%. However, currently the highest performing devices are achieving efficiencies far below this limit with the PCE of record devices at around 8\%, which is also much below that of similar CIGS technology. Low open circuit voltage (V$_{OC}$) relative to the band gap has been recognised as a key bottleneck for the performance of CZTS solar cells. Therefore a major component of this study is to investigate possible origins of this deficiency due to the fundamental material properties of CZTS. It is possible that, in spite of the ideal band gap, other fundamental and unavoidable properties of the material may hinder device performance. The second component of this study therefore is to predict the optoelectronic properties of other metal sulfide materials that until now have received little scientific interest for the application of solar cells, to determine if these materials could be candidate absorber materials for highly efficient solar cells.



\end{abstract}

