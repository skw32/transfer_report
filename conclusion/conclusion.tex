
\chapter{Conclusion \& Future Work}
\label{ch:conclusions}

\begin{itemize}
\item Further work on S vacancies? (discuss defect formation E as a func of S chemical potential and charge corrections for charged defect states)
\item More properties of sulfosalt materials (explicitly discuss SLME approach of predicting PV efficiency from dielectric function) + discuss calculations to predict FE?
\item Many possible extensions with eris:
\item Further investigations with eris? Cu-poor CZTS?
\item Extend eris Hamiltonian to next nearest neighbour interactions (likely to be important for long ranged coulomb interactions)
\item Simulate larger systems with eris - domain formation and band gap fluctuations?
\item Next major project:
\item Simulations to determine most likely surface terminations to inform exptl strategies for surface passivation?
\item QM/MM surface defect simulations? - mention collaboration with PVTEAM?
\end{itemize}


\section{Thermodynamic Disorder of Cu and Zn in {\CZTS}}

\section{Further Development of Monte Carlo Disorder Models}

\section{Predicting the Photovoltaic Performance of Metal Sulfide Materials}
Motivations for sulfosalts (polar materials), show band structures and discuss SLME as future work.

\section{Formation Energy of Sulfur Vacancies in Metal Sulfides}
CZTS work so far and future work for sulfosalts - motivation is volatility of sulfur and particular synthesis methods makes this defect quite likely?

\section{Intrinsic Band Gap Broadening in {\CZTS}}
See commented out vibrational properties section



 